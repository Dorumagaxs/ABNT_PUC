\documentclass[12pt]{article}

\usepackage[utf8]{inputenc}
\usepackage[left=3cm,top=3cm,right=2cm,bottom=2cm,twoside]{geometry}
\usepackage{helvet}
\usepackage{setspace}
\usepackage{tocloft}
\usepackage{titlesec}
\usepackage{fancyhdr}
\usepackage[
	backend=biber,
	bibstyle=referencias,
	maxnames=3,
	minnames=1
]{biblatex}

\addbibresource{referencias.bib}

\usepackage{blindtext}

% Nome da instituição onde o trabalho foi produzido
\newcommand{\instituicao}{Pontifícia Universidade Católica de Minas Gerais}

% Nome do autor, com iniciais maiusculas
% No caso de mais de um autor, manter em ordem alfabética e separar com duas barras invertidas (\\)
\newcommand{\autor}{Autor} % {Autor1\\Autor2}

% Título. Caso haja subtítulo, terminar com dois pontos (:)
\newcommand{\titulo}{Título:}

% Subtítulo. Caso não haja, deixar vazio
\newcommand{\subtitulo}{Subtítulo}

% Número do volume. Se não houver mais de um, deixar vazio
\newcommand{\numerovolume}{}

% Nome da cidade onde  o trabalho foi produzido
\newcommand{\cidade}{Cidade}

% Ano em que o trabalho foi ou será entregue
\newcommand{\ano}{Ano}

\newcommand{\descricao}{
	Trabalho acadêmico apresentado à disciplina Sociologia da faculdade de Direito da {\instituicao}.
}

\newcommand{\professor}{Prof. Dr. professor}


\newcommand{\skippage}{
	\newpage
	\hfill
	\newpage
}

% Definindo Capa
\newcommand{\capa}{
	\begin{center}
		\MakeUppercase{\instituicao}

		\vfill

		\autor

		\vfill

		\MakeUppercase{\textbf{\titulo}}

		\textbf{\subtitulo}

		\vfill

		\numerovolume

		\vfill

		\cidade

		\ano

		\skippage
	\end{center}
}

% Definindo Folha de Rosto
\newcommand{\folhaderosto}{
	\begin{center}
		\autor

		\vfill

		\MakeUppercase{\textbf{\titulo}}

		\textbf{\subtitulo}

		\vspace{1em}

		\numerovolume

		\vspace{1em}

		\hfill
		\begin{minipage}[t]{.6\textwidth}
			\singlespacing
			\descricao

			{\bigskip}Professor: {\professor}
		\end{minipage}

		\vfill

		\cidade

		\ano

		\skippage
	\end{center}
}

% Definindo Resumo
\newcommand{\resumo}[2]{
	\begin{center}
		\MakeUppercase{\textbf{#1}}
	\end{center}

	#2

	\skippage
}

% Definindo Sumário
\AtBeginDocument{\addtocontents{toc}{\protect\thispagestyle{empty}}}
\renewcommand{\contentsname}{
	\begin{center}
		\textbf{SUMÁRIO}
	\end{center}
}
\renewcommand{\cftbeforetoctitleskip}{0em}
\renewcommand{\cftsecdotsep}{\cftdotsep}
\renewcommand{\cftsubsecfont}{\bfseries}
\renewcommand{\cftsubsubsecfont}{\itshape}

% Definindo Capítulo
\newcommand{\capitulo}[1]{
	\newpage
	\ifodd \value{page}
	\else
		\hfill
		\newpage
	\fi
	\section{\MakeUppercase{#1}}
	\indent
}
\titleformat{\subsection}{\LARGE\bfseries}{\thesubsection}{.5em}{}
\titleformat{\subsubsection}{\Large\itshape}{\thesubsubsection}{.5em}{}

% Definindo numeração
\fancyhead{}
\fancyfoot{}
\fancyhead[ORH]{\thepage}
\fancyhead[ELH]{\thepage}
\renewcommand{\headrule}{}

\begin{document}
	\pagestyle{empty}
	\onehalfspacing

	\capa

	\pagenumbering{arabic}
	\folhaderosto

	\resumo{resumo}{% Apresentação sucinta dos pontos relevantes de um trabalho acadêmico (teses, dissertações, especializações e TCC) em um parágrafo único. O resumo deve conter o objeto de estudo, objetivo, pressupostos teóricos, metodologia e resultados. Deve-se usar o verbo na voz ativa e na terceira pessoa do singular e não deve conter nomes de autores e obras. Logo abaixo do resumo, devem-se pontuar as palavras-chave que representam o conteúdo do estudo. São separadas entre si com ponto final e finalizadas também com ponto final. Quanto à sua extensão, os resumos devem ter:

%a) de 150 a 500 palavras, para trabalhos acadêmicos (teses, dissertações, entre outros) e relatórios técnicos-científicos;

%b) de 100 a 250 palavras para artigos de periódicos;

%c) de 50 a 100 palavras para os destinados a indicações breves;

%d) Os resumos críticos não têm limite de palavras.

\lipsum
}
	\resumo{abstract}{% Apresentação sucinta dos pontos relevantes de um trabalho acadêmico (teses, dissertações, especializações e TCC) em um parágrafo único. O resumo deve conter o objeto de estudo, objetivo, pressupostos teóricos, metodologia e resultados. Deve-se usar o verbo na voz ativa e na terceira pessoa do singular e não deve conter nomes de autores e obras. Logo abaixo do resumo, devem-se pontuar as palavras-chave que representam o conteúdo do estudo. São separadas entre si com ponto final e finalizadas também com ponto final. Quanto à sua extensão, os resumos devem ter:

%a) de 150 a 500 palavras, para trabalhos acadêmicos (teses, dissertações, entre outros) e relatórios técnicos-científicos;

%b) de 100 a 250 palavras para artigos de periódicos;

%c) de 50 a 100 palavras para os destinados a indicações breves;

%d) Os resumos críticos não têm limite de palavras.

\lipsum
}

	\tableofcontents
	\newpage

	\chapter{Introdução}
\pagestyle{fancy}

% Tudo abaixo desta linha não passa de texto exemplo. Substitua-o por sua introdução

book_normal $=>$ sad\footnote{bla} bla~\cite[p. 51]{livroo}

site_normal $=>$ bla~\cite{sitee}

book_text $=>$ bla~\textcite[p. 51]{livroo}

site_text $=>$ bla~\textcite[p. 51]{sitee}

book_foot $=>$ bla~\footcite{livroo}

site_foot $=>$ bla~\footcite{sitee}


\lipsum

	\capitulo{capitulo 2}\blindtext

\capitulo{capitulo 3}\blindtext

\capitulo{capitulo 4}\blindtext

\capitulo{capitulo 5}\blindtext

\capitulo{capitulo 6}\blindtext


	\chapter{Conclusão}

% Tudo abaixo desta linha não passa de texto exemplo. Substitua-o por sua conclusão

\lipsum

asd\cite{codigo_civil}

asd\cite{leii}


	\newpage
	\printbibliography[heading=none]
\end{document}

% Apresentação sucinta dos pontos relevantes de um trabalho acadêmico (teses, dissertações, especializações e TCC) em um parágrafo único. O resumo deve conter o objeto de estudo, objetivo, pressupostos teóricos, metodologia e resultados. Deve-se usar o verbo na voz ativa e na terceira pessoa do singular e não deve conter nomes de autores e obras. Logo abaixo do resumo, devem-se pontuar as palavras-chave que representam o conteúdo do estudo. São separadas entre si com ponto final e finalizadas também com ponto final. Quanto à sua extensão, os resumos devem ter:

%a) de 150 a 500 palavras, para trabalhos acadêmicos (teses, dissertações, entre outros) e relatórios técnicos-científicos;

%b) de 100 a 250 palavras para artigos de periódicos;

%c) de 50 a 100 palavras para os destinados a indicações breves;

%d) Os resumos críticos não têm limite de palavras.

\lipsum
